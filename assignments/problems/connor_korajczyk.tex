\citet{connor1986performance} develop a method to measure the performance of a mutual fund portfolio. Using a large number of funds, they show that the \citet{treynor1973use} Appraisal Ratio completely captures investors ranking over funds. A simplified version of the model is described as follows:

Let fund's \(j\) excess return at time \(t\) be denoted as \(r_{j,t}\) for \(j = 1, \dots, N\) and the riskless asset return be \(r_{0,t}\). These returns are generated by a finite set of latent factors \(\mb{f_t} = \bp{f_t^{(1)}, \dots, f_t^{(K)}}\) where \(K\) is known. The return of fund \(j\) is then given by:
\begin{equation}
    \label{eq:fund_returns}
    r_{j,t} = \alpha_j + \sum_{k = 1}^K\beta_j^{(k)}\bp{\gamma^{(k)} + f_t^{(k)}}    + \varepsilon_{j,t}
\end{equation}
with \(\E{\bf{f_t}} = 0, \E{\bf{f_tf_t}^\prime} = I_K\) and \(\E{\varepsilon_{j,t}} = 0\), \(\E{\varepsilon_{j,t}\varepsilon_{j,t}^\prime} = \sigma_j^2\), with \(\sigma_j\) bounded from below. We also assume that \(\varepsilon_{\cdot, t}\) is uncorrelated between funds and independent on the set of factors.  
The Appraisal Ratio is defined as \(t_j = \alpha_j / \sigma_j\), and according to Theorem 4, this ratio is sufficient to rank funds\footnote{i.e, an investor would strictly prefer to switch from investing in fund \(j\) to fund \(l\) if, and only if, \(t_j > t_l\)}.

Let \(\bf{R}\) be the \(N \times T\) matrix of fund returns. The authors show that equation \eqref{eq:fund_returns} can be estimated by the following algorithm.
\begin{enumerate}[label = \arabic*), leftmargin = *]
    \item Compute a \(k \times T\) principal component matrix of \(\frac{1}{n} \bf{R}^\prime \bf{R}\), denote it as \(\wh{\bf{f}}\);
    \item Run a time series regression 
    \begin{equation}
        \label{eq:ck_regression}
        r_j = \wh\alpha_j + \sum_{k = 1}^K\wh\beta_j^{(k)}\bp{\wh\gamma^{(k)} + \wh f^{(k)}} + \wh\varepsilon_j
    \end{equation}
    \item Calculate \(\wh{t_j} = \wh\alpha_j / \wh\sigma_j\).
    \item For a large N and T, this estimator is consistent and asymptotically normal:
    \begin{equation}
        \label{eq:appraisal_distribution}
        \underset{T \to \infty}{\text{dlim}}\hspace{5pt}\underset{N \to \infty}{\text{plim}}\hspace{5pt} T^{1/2}\bp{\wh{t_j} - t_j} \sim N\bp{0, 1 + \sum_{k=1}^K \gamma_k^2}
    \end{equation}
\end{enumerate}

In this exercise, we will apply this model to a dataset of mutual funds. The dataset \texttt{fund\_returns.csv} available on CANVAS contains a time series of monthly log-returns for a large set of mutual funds from 2000 to 2023 obtained from CRSP.\footnote{This data is available on WRDS under the \texttt{MONTHLY\_RETURNS} dataset on the \texttt{CRSP} library.}. We will also use the data on the Fama-French factors and risk-free rate available on the \texttt{ffdaily.csv} file.

\begin{enumerate}[label = \Alph*)]
    \item For \(k\) from 1 to 8, calculate the principal components of the funds' excess returns. Plot the explained variance ratio for each of these components. How do these PC compare with the Fama-French factors?
    \item \citet{connor1993test} provide a methodology to choose the optimal number of factors \(K\) in an approximate factor model. Their test is based on the idea that additional (non-informative) factors will not increase the explained variance of the model. Their algorithm is described as the following iteration:
    \begin{enumerate}[label = \arabic*), leftmargin = *]
        \item For a given number of factors \(k\), estimate the model with \(k\) and \(k+1\) factors. Let the residuals be \(\wh\varepsilon_{j,t}\) and \(\wh\varepsilon^\ast_{j,t}\), respectively;
        \item Calculate adjusted squared residuals
        \begin{equation}
            \begin{aligned}
                \wh\sigma_{j,t} & = \wh\varepsilon_{j,t}/\bs{1-(k+1)/T-k/N} \\
                \wh\sigma_{j,t}^\ast & = \wh\varepsilon_{j,t}^\ast/\bs{1-(k+2)/T-k/N}
            \end{aligned}
        \end{equation}
        \item Calculate \(\mb{\wh\Delta}\) by subtracting the cross-sectional means of \(\wh\sigma_{j,t}\) in odd periods from the cross-sectional means of \(\wh\sigma_{j, t+1}^\ast\) in even periods:
        \begin{equation}
            \wh\Delta_s = \mu_{2s-1} - \mu_{2s}^\ast \qquad s = 1, \dots, \floor{T/2}
        \end{equation}
        where \(\mu_t = N^{-1}\sum_{j=1}^N\wh\sigma_{j,t}^2\) and \(\mu_t^\ast = N^{-1}\sum_{j=1}^N\bp{\wh\sigma_{j,t}^\ast}^2\). Under the null,
        \begin{equation}
            \underset{N \to \infty}{\text{dlim}}\hspace{5pt}\sqrt{N}\wh\Delta \sim N\bp{0, \Gamma}
        \end{equation}
        \item Using the time series of \(\wh\Delta\), calculate the mean and covariance matrix \(\wh\Gamma\) and perform a one-sided zero mean test: \[\Hc_0: \E\bs{\wh\Delta} \leq 0\]
        \item If the null is rejected, start over with \(k + 1\) factors. Otherwise, set \(K = k\) as the optimal number of factors.
    \end{enumerate}
    
    Using their test, find the optimal \(K\). Use this number for the rest of the exercise. 
    \item\label{item:estimate} Estimate \eqref{eq:ck_regression} for each fund and calculate their Appraisal Ratio. Plot the distribution of this measure. According to this, how many funds do really outperform the market? Use the distribution on \eqref{eq:appraisal_distribution} to test the significance of their performance.
    \item Equation \eqref{eq:fund_returns} assumes that the factor loadings \(\beta_j^{(k)}\) are constant over time. Split the sample in half. Redo \ref{item:estimate} for each subsample and compare the results. How many funds do outperform in both periods? Interpret.
\end{enumerate}