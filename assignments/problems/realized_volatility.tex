The objective of this assignment is for you to gain some familiarity with basic features of the high-frequency return data. You should use the data posted on the canvas site for the S\& P 500 ETF labelled the ``spider'' or SPY. It is available both as a csv and mat formatted file.
\begin{enumerate}[label = \arabic*)]
    \item Construct the average trading day realized volatility (RV) across all trading days in the full sample for a variety of underlying sampling frequencies. That is, you should cumulate the squared returns from the market open to market close and average them across all the trading days. This becomes a measure of the average (unconditional) trading-day return variation across the full sample. A reasonable choice is to compute the measures for the 1-min, 2-min, 3-min, 5-min, 7-min, 10-min, 15-min, 20-min, 25-min, 30-min, 45-min, and 60-min frequency. If the frequency does not ``fit'' so there is a small interval left at the end of the trading day, please just add the squared return for this period to the others for the trading day, i.e., if 5 minutes are left over when sampling each 7-min, then the last term is a squared 5-minute return. Plot the values obtained for the average daily trading day RV as a function of the sampling frequency (1-min up to 60-min). This becomes a ``vol signature plot''. Does the plot have an approximately ``flat'' region across some frequencies?
    \item\label{item:abs_ret} For each one-minute return across the trading day, please compute the average absolute return across all trading days in the sample for this interval, and repeat for all one-minute intervals in the trading day. Plot the value against the time-of-day from the open until the close of the trading day. This figure represents the intraday volatility pattern.
    \item Compute the measures in question \ref{item:abs_ret} for each of the first two full years of the sample and for the last two full years of the sample. Please plot them as indicated in question \ref{item:abs_ret}.
    \item Plot the time series of trading day RV measures for the full sample. Do this both for the regular RV measure (cumulative squared returns), but also for the square-root of the trading day RV, and the logarithm of the trading day RV. Compute the auto-correlogram for lags 1 through 20 of these series.
\end{enumerate}