For this question, we are going to use the dataset \texttt{stocks\_pca.zip} available on CANVAS. This dataset contains the daily price of the current constituents of the S\&P 500 index from 2004 to 2023 available on CRSP. Stocks are identifies by their unique PERMNO. Using this data, calculate daily log-returns and drop stocks that contain any missing values in the sample.

Additionally, we are going to use the Fama-French 5 factors + Momentum (FF) dataset available on CANVAS under the \texttt{ffdaily.csv} file. This dataset contains the daily log-returns of the Fama-French 5 factors + Momentum from 2004 to 2023.

\begin{enumerate}[label = \textbf{\Alph*)}]
    \item We start by making some simple Principal Component Analysis (PCA). Extract the first principal component of the log-returns (remember to standardize the returns). How does it look like on the time series? How much of the variance is explained by the first principal component?
    \item Now extend this to up to 8 factors. How much does the explained variable increase with the addition of new factors? Based on this, how many factors would you choose?
    \item \citet{horn1965rationale} provides the ``latent-root-one'' rule to choose the number of factors. According to this rule, the optimal number of factors \(k\) is given by the number of eigenvalues of the correlation matrix that are greater than 1. Using the standardized returns, calculate their correlation matrix and plot the eigenvalues. Following the latent-root-one rule, how many factors would you choose? How does it relates to the \textit{Factor Zoo} problem?
    \item Pick the model with 8 factors. Using rotation, how do the principal components correlate with the FF factors? Orthogonalize the FF factors and the PC and calculate their correlation. How does each PC correlate with the FF factors?
    \item We now turn to using the Canonical Correlation Analysis (CCA) to analyze the relationship between the FF factors and the PCA factors. Using 1 to 5 components, calculate the canonical correlation between these two sets of factors. How does it change with the number of components?
    \item Now obtain the canonical variables by rotating the PCA and FF factors according to the canonical loadings. How much correlation is there between the canonical variables?
\end{enumerate}