\documentclass[12pt,twoside]{article}

%%%%%%% Loading packages and macros %%%%%%%
\usepackage{preamble/__packages__}
\usepackage{preamble/__theme__}
%%%%%%%%%%%%%%%%%%%%%%%%%%%%%%%%%%%%%%%%%%%

%%%%%%% Document %%%%%%%
\newcommand{\assignment}{Assignment 3}
\newcommand{\course}{FINC 585-3}
\newcommand{\prof}{Torben G. Andersen \& Zhengyang Jiang}
\newcommand{\institute}{Kellogg School of Management}

\title{\course-\assignment}
\author{}
\date{Spring 2024}
%%%%%%%%%%%%%%%%%%%%%%%%

\begin{document}

% \nocite{*}
\maketitle

% \problem[Roll Model]

% Provide a proof of the Roll estimator for the bid-ask spread based on the return autocorrelation. Using the original paper \citep{roll1984simple}, it should be straightforward to pin down assumptions that will allow for a proof of the results.

\problem

For this problem, we will be using the dataset \texttt{Shiller\_ie\_Data.xls} available on CANVAS, downloaded from \href{http://www.econ.yale.edu/~shiller/data.htm}{Robert Shiller's website}. It has a pre-calculated \textbf{PE} ratio and provides both \textbf{dividend} and \textbf{price level} data for the S\&P index. The variable construction can be seen from the Excel formulas generating the scaled data from the raw observations.\footnote{For more information on the dataset, check the `Data Description.html' also available on CANVAS.} As indicated, \textbf{T-bill} data are available from many sources including FRED. \par
For this exercise, we take an initial look at predictive return regressions, where macro-finance variables are used to forecast the future equity premium. The task is to gauge whether there is (statistically significant) evidence of predictability in the S\&P equity-index returns minus the risk-free rate (use 3-month T-bill rates – available from FRED at the St. Louis Fed). \par
In formal notation, \(\mathbf{r^e(t+1)=r(t+1)-i(t)}\), where \(r(t+1)\) is the monthly continuously compounded nominal return on the S\&P index from the end of month \(t\) to end of month \(t+1\), and \(i(t)\) is the 3-month T-bill rate at the end of month \(t\). \par
As predictors, we will choose the PE-ratio (price/earnings ratio), the DP-ratio (dividend/price ratio), and the relative interest rate (current 3-month T-bill rate minus the average 3-month T-bill rate over the prior last 12 month). This is labeled \textbf{RREL}. \par
The relevant predictive OLS regression take the form:
\[
    r^e(t+1) = a + b^\prime X(t) + u(t), \hspace{25pt} t = 0,1,\dots,T
\]
where \(t\) refers to the (end of) month for the observation.

\begin{enumerate}[label = \alph*)]
    \item Run the regression above with only the PE-ratio as the explanatory (\(X\)) variable. Use the sample period from January 1963 - December 2022. Obtain OLS standard errors, assuming no heteroskedasticity or autocorrelation in the innovations, and assess the significance of the relevant regression coefficient and compute the \(R^2\) statistic.
    \item Repeat the exercise from a), but with only the DP-ratio as the explanatory variable.
    \item Repeat the exercise from a), but with only the RREL as the explanatory variable.
    \item Repeat the exercise from a), but use both the PE-ratio and the RREL as regressors.
    \item\label{item:hc0} Now, assess the significance of the regression coefficient(s) in the above regressions using heteroscedasticity robust standard errors (White standard errors).
    \item Repeat \ref{item:hc0} with heteroscedasticity and autocorrelation consistent (Newey-West) st. errors.
    \item Does the change of method for computing standard errors make any difference to the size of the estimated standard errors? Does it change your conclusions?
\end{enumerate}

% \clearpage
% \bibliographystyle{plainnat}
% \bibliography{preamble/__references__}
\end{document}